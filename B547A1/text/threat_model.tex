\chapter{Threat Model}
\label{ch:threatmodel}
In this chapter we detail our threat modelling approach, document our iterative
process which iterates the threats, identifies mitigations and manages risk.
Our threat model is an iterative model.  Our approach is as follows:
\begin{enumerate}
    \item{Individually review system architecture and dataflows and identify any
potential threats using the STRIDE methodoly augmented by the elevation of
privilege game cards shown in figure~\ref{fig:spoofcard} through
figure~\ref{fig:eopcard}.}
    \item{As a group, discuss any identified threats and gain concurrence and
understanding of the threat.}
    \item{Assign a bug number to the threat.}
    \item{Decide whether or not to accept this threat as a system risk.  Note
this step is intended to \emph{group} like threats and to allow for a more
detailed assessment of the risk presented by the threat to allow for better
triage of threats given limited time and budgets.}
    \item{Perform a threat risk likelihood vs impact for all accepted risks
using a \emph{modified} OWASP risk methodology.}
    \item{Review mitigations and if necessary iterate on high risk threats.}
\end{enumerate}


\section{Adversary}
\label{sec:adversary}
In order to make secure systems, we must always consider the system design and
threat given a specified adversary.  For our system, our adversary is a moderate
capability Blackhat hacker who seeks to compromise our password system for
financial gain.

\section{Assumptions}
\label{sec:assumptions}
As part of our threat model, we identified several security assumptions which
are documented in \nameref{ch:a1}.  Several assumptions demand deeper discussion.
\begin{enumerate}
    \item{\emph{The OS is secure.} While this may seems trivial to note this
assumption, a fundamental premise of the password manager design is that the OS
functions leveraged by the device such as USB load and clipboard have not been
compromised.  If these key OS functions are compromised then any data in transit
will be vulnerable to disclosure.  We minimize our attack surface by using
volatile memory only and using the USB as our sole source of cryptographic
functionality to the maximum extent possible.}
    \item{\emph{The user will physically protect the device.} In our scenario
analysis, permanent denial of service will result from loss of the device.
Since we do not provide any backup (in order to minimize the attack surface), it
is the user's responsibility to have a disaster plan in case of physical loss.
We are quite confident that passwords will not be compromised in the event of a
loss, but a loss also means that a user will no longer have access to the
password store.}
    \item{\emph{Our adversary is looking for financial gain.} Additionally, our
adversary has the resources to employ moderate BlackHat capabilities but will
not engage in any attacks where the attack cost will exceed the expected
financial gain.}\sidenote[][]{This means that given our measly financial assets,
most attackers will not employ national level attack against our password
manager just to confirm how poor we really are!}  These assumptions are pretty
straight forward.
\end{enumerate}


\section{Threats}
\label{sec:threats}

\subsection{Spoofing Threats}

\begin{marginfigure}%
\centering
  \includegraphics[width=0.4\linewidth]{spoofcard}
  \caption{Spoofing Card from the Elevation of Privilege Game}
  \label{fig:spoofcard}
\end{marginfigure}

In this section we identify the threat to each major element of our system,
usually by using a dataflow diagram.  We conduct a STRIDE analysis on each major
data flow or component using the Microsoft Elevation of Privilege game and
personal inspection.\sidenote{The elevation of privilege game can be found at
\url{https://www.microsoft.com/en-us/SDL/adopt/eop.aspx}} For each threat we
decide whether or not to add it to our risk chart; for similar threats or
threats that depend on a similar attack tree we enter only one threat into the
risk table but track the remaining threats in the bug table to ensure they are
properly managed and mitigated. As we iterated threats, we noted several
limitations of the OWASP Risk Management Methodology:
\begin{enumerate}
    \item{The OWASP model places considerable emphasis on non-repudiation.  For
our analysis, many of the threats can be realized by simple possession of the
USB device which does not require logging of any sort.  Following the risk
methodology model without any modification will yield the highest assessment
(likelihood) for detectability. In the future we would consider amending the
rubric for this particular likelihood factor.}
    \begin{marginfigure}%
    \centering
    \includegraphics[width=0.4\linewidth]{tampercard}
    \caption{Tamper Card from the Elevation of Privilege Game}
    \label{fig:tampercardcard}
    \end{marginfigure}


    \item{The technical impact categories and rubrics were very helpful, however
since many of these attacks were aimed at obtaining the USB password, they rated
artificially high because the cascade of compromise that resulted in obtaining a
password. The rubric for either technical impact or likelihood did not allow for
consideration of existing countermeasures or design features in the device.
Again, this resulted in artificially high risk assessments.  For a second
iteration, we would recommend including an assessment of design features and
countermeasures intended to address the threat or related threat as a better way
to better estimate the likelihood of a risk.}

    \begin{marginfigure}%
    \centering
    \includegraphics[width=0.4\linewidth]{repudcard}
    \caption{Repudiation Card from the Elevation of Privilege Game}
    \label{fig:repudcard}
    \end{marginfigure}

    \item{The business impact categories while useful in a real world exercise
were not applicable to this threat model.  In the future we would look at
substituting more applicable rubrics in place of the existing business impact
factors of the OWASP model.}
\end{enumerate}

In spite of the limitations we discuss, the OWASP model was very useful in
drawing our attention to the highest risk threats.  We benefited from using the
5x5 risk matrix as opposed to the 3x3 OWASP risk matrix because we were able to
break out threats and risk to a finer degree. Spoofing threats deal with an
attacker's ability to masquerade as an individual, process or some other entity
in order to gain access to data and systems.  During our first iteration we
identified several spoofing threats.

\begin{table*}[ht]
    \centering
    \includegraphics[width=\linewidth]{spoof_sum}
    \caption{Summary of Spoofing Threats Found in First Threat Iteration}
    \label{tab:spoofsum}
\end{table*}
We identified 5 threats and adjudicated them. Each threat received a bug id
whether or not they were entered into the risk tables.  Threats that were
similar to other evaluated threats were not accepted into the risk table as
noted. Several spoofing threats were elements in a multi-type threat such as
spoofing resulting in information disclosure. When possible, we identified
potential mitigations for each threat which informs the bug list which is where
the majority of system development is drawn.

\subsection{Tampering Threats}

We identified 10 tampering threats and accepted 5 into the risk tables. Threat ID 004,
``Malware Infested Update'' is a significant threat and represents a category of
several different threats which are intended to compromise the USB device by
delivering malware which can alter the reset counter, keys, and data within the
USB device. To explore the details of this attack vector, we provide
an attack tree as appendix~\nameref{ch:a4}.  Our tampering threats and
potential mitigations are depicted in
table~\ref{tab:tampersum}.

\begin{table*}[]
    \centering
    \includegraphics[width=\linewidth]{tamper_sum}
    \caption{Summary of Tampering Threats Found in First Threat Iteration}
    \label{tab:tampersum}
\end{table*}

\begin{marginfigure}[0.25in]%
\centering
  \includegraphics[width=0.4\linewidth]{infodisclcard}
  \caption{Information Disclosure Card from the Elevation of Privilege Game}
  \label{fig:spoofcard}
\end{marginfigure}

\subsection{Repudiation Threats}


We identified 12 potential repudiation threats.  The webserver used to store and present software and firmware updates
represents the most significant repudiation attack surface.  Compromise of this
element of the password manager can allow an attacker to post tampered software
and firmware files which can result in a full compromise of the password manager
and its contents.
\begin{table*}[ht]
    \centering
    \includegraphics[width=\linewidth]{repud_sum}
    \caption{Summary of Repudiation Threats Found in First Threat Iteration}
    \label{tab:repudsum}
\end{table*}



\subsection{Information Disclosure Threats}
We discovered several information disclosure threats during our analysis.  Many
of these threats where aimed at discovering passwords and keys while one
particular threat was aimed at fuzzing the USB to discover any potential leaks
caused by incorrect input.  A particularly nefarious information disclosure threat
was focused on recovering keys from a wiped device that had been transferred to
another user.

\begin{table*}[ht]
    \centering
    \includegraphics[width=\linewidth]{info_sum}
    \caption{Summary of Information Disclosure Threats Found in First Threat Iteration}
    \label{tab:infosum}
\end{table*}


\subsection{Denial Of Service Threats}
Denial of service for our password manager occurs when either the device is
cleared or no longer in the possession of its owner.  Loss of the USB device
represents a considerable denial of service threat because no backup features
are provisioned.
\begin{table*}[ht]
    \centering
    \includegraphics[width=\linewidth]{dos_sum}
    \caption{Summary of Denial of Service Threats Found in First Threat Iteration}
    \label{tab:dossum}
\end{table*}

Loss of control of the USB which results in an attacker possessing the device
may allow for the opportunity for other more sophisticated attacks, or quite
simply an attack that wipes the device because an attacker entered the wrong
password too many times.





\subsection{Elevation of Privilege Threats}
We identified three elevation of privilege threats, one associated with fuzzing
the USB to discovery entry points, another attack that interrupts the device
reset procedure which retains the old password during a device reset and finally a malformed request to the webserver database.  Elevation
of privilege threat are summarized in table~\ref{tab:eopsum}.


\begin{table*}[ht]
    \centering
    \includegraphics[width=\linewidth]{eop_sum}
    \caption{Summary of Elevation of Privilege Threats Found in First Threat Iteration}
    \label{tab:eopsum}
\end{table*}



\section{Threat Summary by Highest Risk}
Following the OWASP risk mitigation methodology, we accepted 19 risks associated with the threats that we identified.  The full risks, risk tables and methodologies are
summarized in \nameref{ch:a3}. We identified 8 category 2 (\emph{\textbf{high risk}})
threats which represents a significant majority of our risks.  A significant
portion of these threats were associated with compromise of of the password
(impact: 9) despite relatively low likelihood scores.  A summary of our threats
and their risk categories can be found in \nameref{ch:a3}, table~\ref{tab:risksum}.

\begin{marginfigure}%
\centering
  \includegraphics[width=0.4\linewidth]{doscard}
  \caption{Denial of Service Card from the Elevation of Privilege Game}
  \label{fig:doscard}
\end{marginfigure}


\section{Attack Trees}
Several threats were enabled by a similar attack vector.  For example, we
identified multiple threats associated with compromised software and firmware
updates.  Given the strong identification of multiple threats using this vector,
we examined this attack vector using an attack tree.

The attack tree that we have come up with as a good example is compromise the
update of application in the target USB device. The USB device needs to update
its application/software as new features are added and security patches are
required from time to time. Attackers can apply the Man In The Middle (MITM)
attack to the access between the USB device and the web server. The web server
is hosted to provide the new release of the applications. The detailed processes
of this threat is described in the \nameref{ch:a4}.

\begin{marginfigure}%
\centering
  \includegraphics[width=0.4\linewidth]{eopcard}
  \caption{Elevation of Privilege Card from the Elevation of Privilege Game}
  \label{fig:eopcard}
\end{marginfigure}
