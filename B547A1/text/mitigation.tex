\chapter{Mitigations}
\label{ch:mitigations}
In this chapter we describe the mitigations to the threats we found in
the previous chapter.

\section{Mitigation by Threat}
The threat tables detail a proposed mitigation for each identified
threat.  The final implementation is left to the designers who would
address each threat through the bug list. For a product in
development, additional visibility can be afforded by generating
requirements or test cases that identify each threat or risk.

\section{Risk and Prioritization}
\label{sec:risk}
While it is beyond the scope of this assignment to describe any technical or
business processes used to manage risk, we implicitly manage risk through our
assumptions and mitigations.  For example, the risk from the threat of loss of
the USB (resulting in a denial of service) is transferred to the owner of the
device by assuming that the owner will take adequate precautions to prevent loss
or theft of the device.  For the unlikely lucky guess threat, the risk is
accepted because the likelihood of properly guessing a complex password in a
very small, fixed number of tries is unlikely.

\begin{marginfigure}
    \centering
    \includegraphics[width=\linewidth]{riskcats}
    \caption{Risk Categories Used in Threat Modeling}
    \label{fig:riskcats}
\end{marginfigure}

In general, we would expect to pay most attention to priority risks in
increasing order; for example, a category one risk would be addressed before a
category 3 threat risk.

\subsection{Risk Determination}
We used a modified OWASP risk methodology to determine the risks posed by each
threat or group of similar threats. Detailed information concerning the assessed
risks can be found in \nameref{ch:a4}.
