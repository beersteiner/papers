\documentclass[sigconf]{acmart}

\settopmatter{printacmref=false} % Removes citation information below abstract
\renewcommand\footnotetextcopyrightpermission[1]{} % removes footnote with conference information in first column
\pagestyle{plain} % removes running headers

\usepackage{booktabs} % For formal tables

\usepackage{subcaption} % for multiple subfigures
\usepackage{footnote}
\makesavenoteenv{tabular}
\makesavenoteenv{table}


% Copyright
\setcopyright{none}
%\setcopyright{acmcopyright}
%\setcopyright{acmlicensed}
%\setcopyright{rightsretained}
%\setcopyright{usgov}
%\setcopyright{usgovmixed}
%\setcopyright{cagov}
%\setcopyright{cagovmixed}


% DOI
%\acmDOI{10.475/123_4}

% ISBN
%\acmISBN{123-4567-24-567/08/06}

%Conference
%\acmConference[Short Name N/A]{Long Name N/A}{Date N/A}{Location N/A}
%\acmYear{N/A}
%\copyrightyear{2018}


%\acmArticle{N/A}
%\acmPrice{N/A}

% These commands are optional
%\acmBooktitle{Transactions of the ACM Woodstock conference}
%\editor{Jennifer B. Sartor}
%\editor{Theo D'Hondt}
%\editor{Wolfgang De Meuter}


\begin{document}
\title{Learning to extend a Cybersecurity Knowledge Graph}
%\titlenote{Produces the permission block, and
  %copyright information}
%\subtitle{Extended Abstract}
%\subtitlenote{The full version of the author's guide is available as
  %\texttt{acmart.pdf} document}


\author{John Stein}
%\authornote{Dr.~Trovato insisted his name be first.}
%\orcid{1234-5678-9012}
\affiliation{%
  \institution{Indiana University}
  \streetaddress{107 S Indiana Ave}
  \city{Bloomington}
  \state{Indiana}
  \postcode{47405}
}
\email{jodstein@iu.edu}

%\author{David Crandall, PhD}
%%\authornote{The secretary disavows any knowledge of this author's actions.}
%\affiliation{%
%  \institution{Indiana University}
%  \streetaddress{107 S Indiana Ave}
%  \city{Bloomington}
%  \state{Indiana}
%  \postcode{47405}
%}
%\email{djcran@indiana.edu}

% The default list of authors is too long for headers.
%\renewcommand{\shortauthors}{B. Trovato et al.}


\begin{abstract}
The landscape of cybersecurity research and intelligence is vast and complex.  Engineers, developers, and administrators of information technology (IT) systems are faced with the monumental task of knowing about, accessing, making sense of, and making decisions from, cybersecurity knowledge.  \textcolor{red}{For example, …}

Toward that goal, various repositories of cybersecurity knowledge, including some that are machine-readable \textcolor{red}{(refs)}, have been developed and curated by the community.  These efforts have improved the coherence and alignment of interrelated lines of cybersecurity research, and created many opportunities for more powerful machine-aided security analysis.  

Many projects have endeavored to perform meaningful learning tasks aimed at discovery of new cybersecurity knowledge \textcolor{red}{(refs)} or using existing knowledge to automate or augment design and analysis of systems and software \textcolor{red}{(refs)}.  With a few exceptions (SemFuzz, others), semantic aspects of most of these learning tasks are limited, or at least not accessible, from the perspective of external semantic reasoning.  More recently, some of this cybersecurity knowledge has been assembled in the form of knowledge graphs (SEPSES ref) and linked data.  We believe that representation of cybersecurity knowledge as linked entities and relations is a critical step toward enabling higher semantic learning tasks in the cybersecurity domain.

This project seeks to develop a model to expand a cybersecurity knowledge graph generated from structured, manually entered information to include new knowledge from unstructured documents.  When encountering a new, unstructured document, the model (1) identifies cybersecurity-related entities contained in the document, (2) classifies each entity type, and (3) identifies and classifies relations between entities.  The resulting model, therefore, has the capability to propose additions to structured cybersecurity repositories (when it encounters entities or relations that are not currently in the repository) as well as provide graph-based reasoning and context capabilities to new, unstructured information.

\end{abstract}

\keywords{machine learning, models, covert channels, detector}


\maketitle

\section{Introduction}
TBD

\paragraph{Our contributions.} TBD


\section{Methodology}

\subsection{Computing Environment}

TBD

\subsection{Data} 

TBD

\subsection{Models}

TBD

\subsection{Task}
TBD

\section{Results}

TBD


\section{Conclusions}

TBD

\section{Future Work}

TBD


\begin{acks}

This research was supported in part by Lilly Endowment, Inc., through its support for the Indiana University Pervasive Technology Institute. \cite{PTI} % for Carbonate use

TBD - remember compute resources and other boilerplate acks

\end{acks}

\newpage
\appendix
%\newpage
%Appendix A
\section{Appendix A Title} 

TBD


%Appendix B
\section{Appendix B Title}

TBD





\bibliographystyle{ACM-Reference-Format}
\bibliography{bibliography}

\end{document}
