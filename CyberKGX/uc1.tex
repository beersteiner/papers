A software engineer is hired to develop a back-end application for a new Internet of Things (IoT) device.  She is given a software requirements specification, a hardware interface specification, a small budget, and tight schedule.  The developer appreciates the importance of security and privacy, especially in the IoT market, and understands that her design implementation decisions will have significant impact on security and privacy, but she is not a cybersecurity expert and cannot afford to dedicate a major portion of the schedule or budget to security.

\paragraph{Status Quo} Near the beginning of the project, she performs an internet survey of existing popular IoT backend frameworks, narrows down to 2 candidates, and engages in some internet research on pros and cons of each, including security.  The number of security hits returned on each is overwhelming.  Some vulnerabilities relate only to older versions and have since been patched.  Some vulnerabilities manifest from poor implmentation.  Some vulnerabilities seem to be instances of weaknesses found in all candidate frameworks.  Hence, most of the search results are not actionable and the developer simply chooses the framework that best supports other project requirements.  In the best case, she may track documented vulnerabilities associated with the chosen framework and implement mitigations into the system.

\paragraph{With CyberKGX}