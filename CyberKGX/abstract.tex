Security analysis is an inherently complex problem.  Vulnerabilities can manifest at any level of a system, ranging from flaws in logic, primitives, or algorithms, to flaws in the configuration or inappropriate use of otherwise correct components/constituents, to unpredictable and undesirable sytem- or system-of-systems-level behavior not attributable to any of the components/constituents in isolation (emergent).  For ease of reference, we label these manifestations as Class 1, 2, and 3, respectively.  

Traditional cybersecurity research and cyberintelligence approaches endeavor to produce security data (weaknesses, product vulnerabilities, attack patterns, etc.) to enable the broader cybersecurity community to perform informed assessment and decision making within the cyber threat landscape.  These traditional approaches have proven demonstrably successful in producing security data (dozens of new vulnerabilities are reported every day \cite{noauthor_browse_nodate}).  

While these approaches have enabled the development of automation capability to track and manage known vulnerabilities \cite{noauthor_cve_nodate}, they fall short of enabling informed assessment and decision making with respect to Class 2 and Class 3 vulnerabilities \textcolor{red}{(need support for this claim)}.


% Toward that goal, various repositories of structured cybersecurity knowledge \textcolor{red}{(refs)}, have been developed and curated by the community.  These efforts have improved the coherence and alignment of interrelated lines of cybersecurity research, and created many opportunities for more powerful machine-aided security analysis.  

% Many projects have endeavored to perform meaningful learning tasks aimed at discovery of new cybersecurity knowledge \textcolor{red}{(refs)} or using existing knowledge to automate or augment design and analysis of systems and software \textcolor{red}{(refs)}.  With a few exceptions (SemFuzz, others), semantic aspects of most of these learning tasks are limited, or at least not accessible, from the perspective of external semantic reasoning.  More recently, some of this cybersecurity knowledge has been assembled in the form of knowledge graphs (SEPSES ref) and linked data.  We believe that representation of cybersecurity knowledge as linked entities and relations is a critical step toward enabling higher semantic learning tasks in the cybersecurity domain.

% This project seeks to develop a model to expand a cybersecurity knowledge graph generated from structured, manually entered information to include new knowledge from unstructured documents.  When encountering a new, unstructured document, the model (1) identifies cybersecurity-related entities contained in the document, (2) classifies each entity type, and (3) identifies and classifies relations between entities.  The resulting model, therefore, has the capability to propose additions to structured cybersecurity repositories (when it encounters entities or relations that are not currently in the repository) as well as provide graph-based reasoning and context capabilities to new, unstructured information.