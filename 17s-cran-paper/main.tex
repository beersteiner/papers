\documentclass[sigconf]{acmart}

\settopmatter{printacmref=false} % Removes citation information below abstract
\renewcommand\footnotetextcopyrightpermission[1]{} % removes footnote with conference information in first column
\pagestyle{plain} % removes running headers

\usepackage{booktabs} % For formal tables

\usepackage{subcaption} % for multiple subfigures
\usepackage{footnote}
\makesavenoteenv{tabular}
\makesavenoteenv{table}


% Copyright
\setcopyright{none}
%\setcopyright{acmcopyright}
%\setcopyright{acmlicensed}
%\setcopyright{rightsretained}
%\setcopyright{usgov}
%\setcopyright{usgovmixed}
%\setcopyright{cagov}
%\setcopyright{cagovmixed}


% DOI
%\acmDOI{10.475/123_4}

% ISBN
%\acmISBN{123-4567-24-567/08/06}

%Conference
%\acmConference[Short Name N/A]{Long Name N/A}{Date N/A}{Location N/A}
%\acmYear{N/A}
%\copyrightyear{2018}


%\acmArticle{N/A}
%\acmPrice{N/A}

% These commands are optional
%\acmBooktitle{Transactions of the ACM Woodstock conference}
%\editor{Jennifer B. Sartor}
%\editor{Theo D'Hondt}
%\editor{Wolfgang De Meuter}


\begin{document}
\title{A Covert Channel Detector for Machine Learning Models}
%\titlenote{Produces the permission block, and
  %copyright information}
%\subtitle{Extended Abstract}
%\subtitlenote{The full version of the author's guide is available as
  %\texttt{acmart.pdf} document}


\author{John Stein}
%\authornote{Dr.~Trovato insisted his name be first.}
%\orcid{1234-5678-9012}
\affiliation{%
  \institution{Indiana University}
  \streetaddress{107 S Indiana Ave}
  \city{Bloomington}
  \state{Indiana}
  \postcode{47405}
}
\email{jodstein@iu.edu}

%\author{David Crandall, PhD}
%%\authornote{The secretary disavows any knowledge of this author's actions.}
%\affiliation{%
%  \institution{Indiana University}
%  \streetaddress{107 S Indiana Ave}
%  \city{Bloomington}
%  \state{Indiana}
%  \postcode{47405}
%}
%\email{djcran@indiana.edu}

% The default list of authors is too long for headers.
%\renewcommand{\shortauthors}{B. Trovato et al.}


\begin{abstract}
It has been shown that malicious model training code can augment sensitive training data with specially-crafted samples to enable later exfiltration of the sensitive data by an attacker who has black-box access to the completed model, but no direct access to the sensitive training data. \cite{DBLP:journals/corr/abs-1709-07886}  This is known as a Covert Channel for machine learning models.  It has also been shown an attacker can infer whether a specific sample was included in the training set for a black-box model if the attacker has knowledge of the model's structure or can generate models with similar structure. \cite{DBLP:journals/corr/ShokriSS16}  This is known as Membership Inference on machine learning models.  We demonstrate that a variation of the Membership Inference `attack' can be used to detect whether a given machine learning model may have a covert channel.  We review some performance factors of the detector, and discuss how performance is directly related to model capacity and sparsity of the sensitive and maliciously-crafted data.
\end{abstract}

\keywords{machine learning, models, covert channels, detector}


\maketitle

\section{Introduction}
TBD

\paragraph{Our contributions.} TBD


\section{Methodology}

\subsection{Computing Environment}

TBD

\subsection{Data} 

TBD

\subsection{Models}

TBD

\subsection{Task}
TBD

\section{Results}

TBD


\section{Conclusions}

TBD

\section{Future Work}

TBD


\begin{acks}

This research was supported in part by Lilly Endowment, Inc., through its support for the Indiana University Pervasive Technology Institute. \cite{PTI} % for Carbonate use

TBD - remember compute resources and other boilerplate acks

\end{acks}

\newpage
\appendix
%\newpage
%Appendix A
\section{Appendix A Title} 

TBD


%Appendix B
\section{Appendix B Title}

TBD





\bibliographystyle{ACM-Reference-Format}
\bibliography{bibliography}

\end{document}
